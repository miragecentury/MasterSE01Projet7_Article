\documentclass[10pt,a4paper]{article}
\usepackage[utf8]{inputenc}
\usepackage{amsmath}
\usepackage{amsfonts}
\usepackage{amssymb}
\author{COMPIEGNE MAITE VANROYE VICTORIEN}
\title{GEOLOCALISATION, LES NOUVEAUX OBJETS}
\begin{document}
\paragraph{Introduction}
Ici se trouvera l'introduction
\paragraph{Présentation de l'existant}
Ici se trouvera la présentation de l'existant
\paragraph{Sigfox}
est une technologie développé par l'entreprise du même nom. Cette entreprise a fait le parie de miser non pas sur le haut débit mais le bas débit afin de limiter au maximum le coût et la consommation énergétique. Leur but étant de fournir un support de communication accessible à tous pour l'Internet des objets.\\ Le premier point important de cette technologie est la consommation énergétique qui est inférieur au gsm habituel qui comme pour nos téléphones abaissent grandement la durée des batteries. Ici, sur 162305984 Envois Sigfox, on a économisé 4055703 Watt-heure par rapport au GSM (Source: Site Sigfox), le modem consomme 50 micro-watt comparé au 5000 du cellulaire. \\Le second point intéressant est celui de la couverture du réseau qui contrairement au réseau mobile, les antennes SigFox ont une portée bien supérieur et donc facilite le déploiement. Actuellement, le réseau en France n'est pas encore terminé mais le sera prochainement. A l'international, la société a fait une levé de fond pour commencer le déploiement, vous pourrez observer l'avancement sur le site de Sigfox de celui-ci. La grande contrainte actuellement de cette technologie est qu'elle n'est disponible qu'en émission, les modem peuvent envoyer mais pas encore recevoir. L'entreprise Sigfox attend la fin de déploiement du réseau pour mettre en place cette fonctionnalité ce qui viendra compléter parfaitement les capacités.\\Troisième points qui est la contrainte majeur est la taille des messages et le débit qui est de 12 Octets/Messages à 100 Bits/Secondes et de 140 Messages/Jours. Mais cette limitation s'explique par le cadre d'utilisation visé, les objets connectés sont très souvent très spécialisé et sont limité dans leur fonctionnalité, on peut donc voir que leur besoin en débit est très faible, sauf cas de caméra ou d'autre capteur gourmand en donnée, si on prend pour exemple un capteur de température et humidité, il est très facile de faire passer dans les messages ces données et même le niveau de batterie. Et là, encore l'Internet des objets n'est pas idées qui manquent et Sigfox apparait comme l'un des meilleurs outils pour la communication de ceux-ci. Nous allons voir maintenant comment nous avons mis en place cette technologie dans le cadre de carde de localisation.
\paragraph{Présentation des deux design de carte}
TODO
\paragraph{Bilan de consommation des deux designs + une carte existante}

\end{document}